% TREEE white paper

%----------------------------------------------------------------------------------------
%	PACKAGES AND OTHER DOCUMENT CONFIGURATIONS
%----------------------------------------------------------------------------------------

\documentclass[twoside,twocolumn]{article}

\usepackage{blindtext} % Package to generate dummy text throughout this template 

\usepackage[sc]{mathpazo} % Use the Palatino font
\usepackage[T1]{fontenc} % Use 8-bit encoding that has 256 glyphs
%\linespread{1.05} % Line spacing - Palatino needs more space between lines
\usepackage{microtype} % Slightly tweak font spacing for aesthetics
\usepackage{eufrak}
\usepackage{graphicx} % For \scalebox

\usepackage[english]{babel} % Language hyphenation and typographical rules

\usepackage[hmarginratio=1:1,top=32mm,columnsep=20pt]{geometry} % Document margins
\usepackage[hang, small,labelfont=bf,up,textfont=it,up]{caption} % Custom captions under/above floats in tables or figures
\usepackage{booktabs} % Horizontal rules in tables

\usepackage{lettrine} % The lettrine is the first enlarged letter at the beginning of the text

\usepackage{enumitem} % Customized lists
\setlist[itemize]{noitemsep} % Make itemize lists more compact

\usepackage{abstract} % Allows abstract customization
\renewcommand{\abstractnamefont}{\normalfont\bfseries} % Set the "Abstract" text to bold
\renewcommand{\abstracttextfont}{\normalfont\small\itshape} % Set the abstract itself to small italic text

\usepackage{titlesec} % Allows customization of titles
\renewcommand\thesection{\Roman{section}} % Roman numerals for the sections
\renewcommand\thesubsection{\arabic{subsection}} % roman numerals only for subsections
\titleformat{\section}[block]{\Large\scshape\centering}{\thesection.}{1em}{} % Change the look of the section titles
\titleformat{\subsection}[block]{\large\scshape}{\thesubsection.}{1em}{} % Change the look of the section titles

\usepackage{fancyhdr} % Headers and footers
\pagestyle{fancy} % All pages have headers and footers
\fancyhead{} % Blank out the default header
\fancyfoot{} % Blank out the default footer
\fancyhead[C]{Crumbled Data Storage $\bullet$ Cyril Dever} % Custom header text
\fancyfoot[RO,LE]{\thepage} % Custom footer text
\setlength{\headheight}{14pt}

\usepackage{titling} % Customizing the title section

\usepackage{hyperref} % For hyperlinks in the PDF

\usepackage[symbol]{footmisc} % To use special character in footnote
\renewcommand{\thefootnote}{\arabic{footnote}}

\usepackage{outlines}
\usepackage[font=itshape]{quoting}

\usepackage[linesnumbered,ruled,vlined]{algorithm2e}
\SetKw{Continue}{continue}
\SetKw{KwBy}{by}

%----------------------------------------------------------------------------------------
%	FUNCTIONS
%----------------------------------------------------------------------------------------

\newcommand{\ceil}[1]{\left\lceil #1 \right\rceil}
\newcommand{\floor}[1]{\left\lfloor #1 \right\rfloor}
\newcommand{\bsfnote}{\textsuperscript{*}} % for reference to the base64 string note
\newcommand{\hexnote}{\textsuperscript{$\dagger$}} % for reference to the hex string note
\newcommand{\mod}[1]{\ \mathrm{mod}\ #1}

%----------------------------------------------------------------------------------------
%	LOW LEVEL SECTIONS
%----------------------------------------------------------------------------------------

\usepackage{titlesec}

\setcounter{secnumdepth}{4} % eq. subsubsubsection
\titleformat{\paragraph}
{\normalfont\normalsize\bfseries}{\theparagraph}{1em}{}
\titlespacing*{\paragraph}
{0pt}{3.25ex plus 1ex minus .2ex}{1.5ex plus .2ex}

%----------------------------------------------------------------------------------------
%	LISTINGS
%----------------------------------------------------------------------------------------

\usepackage{amsthm}
\theoremstyle{definition}
\newtheorem{definition}{Definition}
\newtheorem{proposition}{Proposition}

\theoremstyle{remark}
\newtheorem*{remark}{Note}
\newtheorem*{recall}{Recall}

%----------------------------------------------------------------------------------------
%	FIGURES
%----------------------------------------------------------------------------------------

\usepackage{tikz}
\usepackage{caption}

\usetikzlibrary{shapes.geometric, arrows, calc, positioning}

\tikzstyle{startstop} = [rectangle, rounded corners, minimum width=3cm, minimum height=1cm,text centered, draw=black]
\tikzstyle{io} = [trapezium, trapezium left angle=70, trapezium right angle=110, minimum width=3cm, minimum height=1cm, text centered, text width=1.7cm, inner sep=0.4cm, draw=black]
\tikzstyle{process} = [rectangle, minimum width=3cm, minimum height=1cm, text centered, draw=black]
\tikzstyle{decision} = [diamond, minimum width=3cm, minimum height=1cm, text centered, inner sep=-0.1cm, draw=black]
\tikzstyle{arrow} = [thick,->,>=stealth]
\tikzset{XOR/.style={draw,circle,append after command={
        [shorten >=\pgflinewidth, shorten <=\pgflinewidth,]
        (\tikzlastnode.north) edge (\tikzlastnode.south)
        (\tikzlastnode.east) edge (\tikzlastnode.west)
        }
    }
}

%----------------------------------------------------------------------------------------
%	TITLE SECTION
%----------------------------------------------------------------------------------------

\usepackage[english]{datetime2}
\DTMsavedate{thedate}{2018-11-26}

\setlength{\droptitle}{-5\baselineskip} % Move the title up

\pretitle{\begin{center}\Large\bfseries}
\posttitle{\end{center}}
\title{Fast Indexing Engine \\for Data Identified by a Hashed ID \\and Stored in an Immutable File} % Title
\author{%
    \textsc{Cyril Dever}\\ % Name
    \normalsize Edgewhere \\ % Institution
}
% \date{\today} % Leave empty to omit a date
\date{\DTMusedate{thedate}}
\renewcommand{\maketitlehookd}{%
    \begin{abstract}
        \noindent We define here an algorithm for indexing the system based on identifiers that are hashed values which is at the same time very 
        powerful to the writing and the reading. We call the Treee\texttrademark.
    \end{abstract}
}

%----------------------------------------------------------------------------------------

\begin{document}

% Print the title
\maketitle

%----------------------------------------------------------------------------------------
%	ARTICLE CONTENTS
%----------------------------------------------------------------------------------------

\section{Introduction}

\lettrine[nindent=0em,lines=3]{T}he challenge was to set up a powerful yet safe search engine to use when the data is some linked list of items that 
could be themselves connected to each other in subchains, indexed through their identifiers that are only made of hashed values (like SHA-256 string 
representations), and all stored in an immutable file.

Its best application is for a blockchain file where an \emph{item} is a transaction embedding a smart contract, and each subchain of items the 
subsequent uses and/or modifications of this smart contract.

This present document describes such indexing engine.

\begin{definition}[Item]
    \label{item}
    An \emph{item} is the actual object recorded in the immutable file subject to indexing.

    As previously stated, it is initially meant to be a transaction or a block in a blockchain file.
\end{definition}

\begin{definition}[Leaf]
    \label{leaf}
    A leaf $\lambda$ embeds the information helping to retrieve one or more possibly linked \emph{items}.
\end{definition}

%----------------------------------------------------------------------------------------

\section{Formal Description}

\subsection{Acyclic graph}

Treee\texttrademark~is an algorithm for indexing \emph{items} recorded in an immutable file based on their identifiers that are hashed values.

It is constructed as an acyclic graph (a tree $T$), each node containing either a node address (its sons) or a set of \emph{leaf}.

\begin{definition}[Hashed Identifier]
    \label{hashedIdentifier}
    We call $\iota$ a hashed identifier (or hashed value) the hexadecimal string representation of the result of a data $d$ passed through a hashing 
    function $\mathfrak{h}()$ such as:
    \begin{equation}
        \label{eq:hashedIdentifier}
        \iota := \mathfrak{h}(d)
    \end{equation}

    The passed data $d$ could be anything but it must be unique if it were to be used as identifier per se.
    
    In blockchains we operate, this data $d$ is usually the \emph{item} itself, ie. a transaction or a block.

    The hashing function could use any cryptographic hashing algorithm as long as it is set beforehand and once and for all in the targeted system
    \footnote{We currently use the \texttt{SHA-256} algorithm because of its wide adoption in both end-user and back-end environments}.
    The number $\mathfrak{N}$ represents the number of bits of the returned hash, eg. $256$ for \texttt{SHA-256}.
\end{definition}

The number of sons of a node is deterministic and depends on the depth of the tree.

Let $p_k$ be the number of sons of a node $N_k$ at depth $k$.

The goal is to create a balanced tree whose width is adaptive to decrease depth and optimize performance.

We are looking to index numbers, in this case the numerical value of the \emph{item}'s unique identifiers $\iota_i (\forall i \in \mathbb{N})$.

\begin{recall}
    An identifier is at its core a hashed value, that is its digest is fundamentally a byte array that could represent any positive integer.
\end{recall}

\subsection{Index}

We now explain the course of the index.

Let $\iota_i^b$ be the value of the hashed identifier for the \emph{item} $i$ in binary form, eg. $$
    \iota_i^b := \texttt{"a1"} \mapsto \texttt{10100001}
$$
indicating its position in the tree $T$.

At each step $j := [0..n) \mid n \leq \mathfrak{N}$, we would pass to child $0$ if the $j$-th bit of $\iota_i^b$ equals $0$; otherwise, we would 
pass to child $1$.

Let $R_j^{\iota_i}$ be the value of this representation of $\iota_i^b$ at step $j < k$.

For a full tree $T$, we build a representation of this number at each step and traverse the tree the same way. 

At the step $j$ of depth $k$, we pass to child $0$ if $R_j^{\iota_i}$ equals $0$, we pass to child $1$ if $R_j^{\iota_i}$ equals $1$, \dots, we 
pass to child $p_{k-1}$ if $R_j^{\iota_i}$ equals $p_{k-1}$.
We stop when the node is an empty leaf $\lambda_j$.

\begin{definition}[Representative]
    \label{representative}
    To construct $R_j^{\iota_i}$, this representative at step $j$, we will successively take the modulo of prime numbers, each step $j$ using the 
    $j$-th prime number in the ordered sequence of all prime numbers $\mathcal{P} := [1, 2, 3, 5, 7, 11, 13, \dots]$, starting at $0$.

    This construction ensures that $R_j^{\iota_i}$ be unique.

    \begin{proof}
        According to the Chinese remainder theorem\cite{gauss}, each number has a unique representative that could be written as the continuation 
        of these modulos.

        Indeed, a number $n$ can be written in the following form:$$
            n \equiv n \mod P_i
        $$
        where $P_i$ is the $i$-th modulo in $\mathcal{P}$.

        Modulos are calculated in $O(1)$ for fixed-sized integers.
        Since the multiplication is faster than the division (necessary for the calculation of the modulo), one may use 
        multiplications by means of floating:$$
            P_i \times \left( n - \floor{n \times \frac{1}{P_i}} \right)
        $$
        This writing in the form of a sequence allows to uniquely define each integer $n$.
    \end{proof}
\end{definition}

Given the random nature of the numbers (or pseudo-random, since the identifiers of the \emph{items} are generated by cryptographic hashing 
technologies), the tree $T$ is balanced.

To unbalance it in a malicious way, it would be necessary to be able to generate hashes whose modulo follows a particular trajectory.

However, the difficulty of such an operation increases exponentially (in the order of $e^{\left(k \times log(k) \right)}$ where $k$ is the depth of 
$T$).

As a reminder, the product of the first 16 prime numbers already equals: $$
    32,589,158,477,190,044,730 \simeq 3 \times 10^{19}
$$
Therefore, as soon as the index contains a reasonably large amount of data, unbalancing the tree in a malicious way would become more and more 
impossible, if at all possible.

\subsection{Leaf is information on item}

Let $s$ be a suchain of linked \emph{items}. For example, it could be a sequence of transactions between two stakeholders defining the progressive 
evolution of the terms of their smart contract.

And let $s_0$ be the first \emph{item} in a subchain of linked \emph{items}.

\vspace{1em} % one line separation

A leaf $\lambda_{i_s} \in T$ contains the following list of information about an \emph{item} $i$ referred to by its identifier $\iota_{i_s}$ in 
subchain $s$:
\begin{itemize}
    \item Identifier (ID) of the current \emph{item} (as a hash string):$$
        \lambda_{i_s}^{ID} := \iota_{i_s}
    $$
    \item Position, ie. start address of the current \emph{item} in the file, eg. $$
        \lambda_{i_s}^{Pos} := 12080
    $$
    \item Size (in bytes) of the saved item in the file, eg. $$
        \lambda_{i_s}^{Size} := 2074
    $$
    \item Origin, ie. the unique identifier of the \emph{item} that is at the start of the \emph{item}'s subchain (if any):$$
        \lambda_{i_s}^{Origin} := \iota_{i_{s_0}}
    $$
    \item Previous, ie. the optional unique identifier of the previous \emph{item} chained to it:$$
        \lambda_{i_s}^{Prev} := \iota_{i_{s-1}}
    $$
    \item Next, ie. the optional unique identifier of the next \emph{item} chained:$$
        \lambda_{i_s}^{Next} := \iota_{i_{s+1}}
    $$
\end{itemize}

A leaf whose next \emph{item} field is empty is the last item in the subchain:
\begin{equation}
    \label{eq:lastItem}
    \lambda_{i_s}^{Next} = \emptyset \iff \iota_{i_{s+1}} \not\in T
\end{equation}

A leaf whose origin \emph{item} field is equal to the identifier of the current \emph{item} is necessarily the origin of the subchain:
\begin{equation}
    \label{eq:firstItem}
    \lambda_{i_s}^{Origin} = \lambda_{i_s}^{ID} \iff s = s_0
\end{equation}

As such, it has a particular operating since, if there were to be one or more \emph{items} thereafter, the last \emph{item} of the subchain will be 
identified here as its previous item. Therefore, let $s_z$ be the last index in a subchain of linked \emph{items}, we have:
\begin{equation}
    \label{eq:circularAtOrigin}
    \left\{
        \begin{array}{l}
            \lambda_{i_s}^{Origin} = \lambda_{i_s}^{ID} \\ \\
            \lambda_{i_s}^{Prev} \neq \emptyset \\
        \end{array}
    \right.
    \iff \lambda_{i_s}^{Prev} := \iota_{i_{s_z}}
\end{equation}

The last three fields of the leaf therefore transforms $s$ as a circular linked list.

%----------------------------------------------------------------------------------------

\section{Implementation}

\subsection{Node creation}

Each step $j$ is paired with the $j$-th prime number in $\mathcal{P}$, eg. the prime number used is $11$ on one step $5$\footnote{We consider step 
$0$ with prime number $1$ being the root of the tree $T$, hence not being counted.}.

At run time, a node is either a parent or a leaf, the latter being an end to a branch in the tree.

\begin{definition}[Parent node]
    \label{parent}
    A parent node is a node containing other nodes, either leaves or other parent nodes.

    A leaf is not a parent node by definition.
\end{definition}

A new parent node must be created every time a representative number walks through the same path as a previous one up to the existing node, 
extending the branch by one step.

For example, at step $k$, if a node contains the leaf for $R_k^{\iota_x}$ for \emph{item} $x$ and if, for a new \emph{item} $y$, $R_k^{\iota_y} = 
R_k^{\iota_x}$, then both \emph{items} $x$ and $y$ will see their leaf move to step $k+1$ (where the two new representatives $R_{k+1}^{\iota_x}$ 
and $R_{k+1}^{\iota_y}$ would give the coordinates of each respective leaf). At step $k$ now lies a new node with two children: the leaf $\lambda_x$ 
and the leaf $\lambda_y$.

Should another \emph{item} $z$ have $R_k^{\iota_z} = R_k^{\iota_y} = R_k^{\iota_x}$ at step $k$, either a third leaf ($\lambda_z$) would be added 
in $k+1$, or the leaf with similar path at $k+1$ would become a new parent node and the two leaves would move to $k+2$. So on and so forth...


%\vfill\eject % To force break column if need be
%\tableofcontents % Uncomment to add a table of contents

%----------------------------------------------------------------------------------------
%	REFERENCE LIST
%----------------------------------------------------------------------------------------

\begin{thebibliography}{99} % Bibliography

\bibitem[1]{gauss}
Gauss. \emph{Disquisitiones Arithmeticae}, translated by Arthur A. Clarke, Springer, 1986.

\end{thebibliography}

%----------------------------------------------------------------------------------------

\end{document}