% TREEE white paper

%----------------------------------------------------------------------------------------
%	PACKAGES AND OTHER DOCUMENT CONFIGURATIONS
%----------------------------------------------------------------------------------------

\documentclass[twoside,twocolumn]{article}

\usepackage{blindtext} % Package to generate dummy text throughout this template 

\usepackage[sc]{mathpazo} % Use the Palatino font
\usepackage[T1]{fontenc} % Use 8-bit encoding that has 256 glyphs
%\linespread{1.05} % Line spacing - Palatino needs more space between lines
\usepackage{microtype} % Slightly tweak font spacing for aesthetics
\usepackage{eufrak}
\usepackage{graphicx} % For \scalebox

\usepackage[english]{babel} % Language hyphenation and typographical rules

\usepackage[hmarginratio=1:1,top=32mm,columnsep=20pt]{geometry} % Document margins
\usepackage[hang, small,labelfont=bf,up,textfont=it,up]{caption} % Custom captions under/above floats in tables or figures
\usepackage{booktabs} % Horizontal rules in tables

\usepackage{lettrine} % The lettrine is the first enlarged letter at the beginning of the text

\usepackage{enumitem} % Customized lists
\setlist[itemize]{noitemsep} % Make itemize lists more compact

\usepackage{abstract} % Allows abstract customization
\renewcommand{\abstractnamefont}{\normalfont\bfseries} % Set the "Abstract" text to bold
\renewcommand{\abstracttextfont}{\normalfont\small\itshape} % Set the abstract itself to small italic text

\usepackage{titlesec} % Allows customization of titles
\renewcommand\thesection{\Roman{section}} % Roman numerals for the sections
\renewcommand\thesubsection{\arabic{subsection}} % roman numerals only for subsections
\titleformat{\section}[block]{\Large\scshape\centering}{\thesection.}{1em}{} % Change the look of the section titles
\titleformat{\subsection}[block]{\large\scshape}{\thesubsection.}{1em}{} % Change the look of the section titles

\usepackage{fancyhdr} % Headers and footers
\pagestyle{fancy} % All pages have headers and footers
\fancyhead{} % Blank out the default header
\fancyfoot{} % Blank out the default footer
\fancyhead[C]{Crumbled Data Storage $\bullet$ Cyril Dever} % Custom header text
\fancyfoot[RO,LE]{\thepage} % Custom footer text
\setlength{\headheight}{14pt}

\usepackage{titling} % Customizing the title section

\usepackage{hyperref} % For hyperlinks in the PDF

\usepackage[symbol]{footmisc} % To use special character in footnote
\renewcommand{\thefootnote}{\arabic{footnote}}

\usepackage{outlines}
\usepackage[font=itshape]{quoting}

\usepackage[linesnumbered,ruled,vlined]{algorithm2e}
\SetKw{Continue}{continue}
\SetKw{KwBy}{by}

%----------------------------------------------------------------------------------------
%	FUNCTIONS
%----------------------------------------------------------------------------------------

\newcommand{\ceil}[1]{\left\lceil #1 \right\rceil}
\newcommand{\floor}[1]{\left\lfloor #1 \right\rfloor}
\newcommand{\bsfnote}{\textsuperscript{*}} % for reference to the base64 string note
\newcommand{\hexnote}{\textsuperscript{$\dagger$}} % for reference to the hex string note
\newcommand{\mod}[1]{\ \mathrm{mod}\ #1}

%----------------------------------------------------------------------------------------
%	LOW LEVEL SECTIONS
%----------------------------------------------------------------------------------------

\usepackage{titlesec}

\setcounter{secnumdepth}{4} % eq. subsubsubsection
\titleformat{\paragraph}
{\normalfont\normalsize\bfseries}{\theparagraph}{1em}{}
\titlespacing*{\paragraph}
{0pt}{3.25ex plus 1ex minus .2ex}{1.5ex plus .2ex}

%----------------------------------------------------------------------------------------
%	LISTINGS
%----------------------------------------------------------------------------------------

\usepackage{amsthm}
\theoremstyle{definition}
\newtheorem{definition}{Definition}
\newtheorem{proposition}{Proposition}

\theoremstyle{remark}
\newtheorem*{remark}{Note}
\newtheorem*{recall}{Recall}

%----------------------------------------------------------------------------------------
%	FIGURES
%----------------------------------------------------------------------------------------

\usepackage{tikz}
\usepackage{caption}

\usetikzlibrary{shapes.geometric, arrows, calc, positioning}

\tikzstyle{startstop} = [rectangle, rounded corners, minimum width=3cm, minimum height=1cm,text centered, draw=black]
\tikzstyle{io} = [trapezium, trapezium left angle=70, trapezium right angle=110, minimum width=3cm, minimum height=1cm, text centered, text width=1.7cm, inner sep=0.4cm, draw=black]
\tikzstyle{process} = [rectangle, minimum width=3cm, minimum height=1cm, text centered, draw=black]
\tikzstyle{decision} = [diamond, minimum width=3cm, minimum height=1cm, text centered, inner sep=-0.1cm, draw=black]
\tikzstyle{arrow} = [thick,->,>=stealth]
\tikzset{XOR/.style={draw,circle,append after command={
        [shorten >=\pgflinewidth, shorten <=\pgflinewidth,]
        (\tikzlastnode.north) edge (\tikzlastnode.south)
        (\tikzlastnode.east) edge (\tikzlastnode.west)
        }
    }
}

%----------------------------------------------------------------------------------------
%	TITLE SECTION
%----------------------------------------------------------------------------------------

\usepackage[english]{datetime2}
\DTMsavedate{thedate}{2018-11-26}

\setlength{\droptitle}{-5\baselineskip} % Move the title up

\pretitle{\begin{center}\Large\bfseries}
\posttitle{\end{center}}
\title{Fast Indexing Engine \\for Data Identified by a Hashed ID \\and Stored in an Immutable File} % Title
\author{%
    \textsc{Cyril Dever}\\ % Name
    \normalsize Edgewhere \\ % Institution
}
% \date{\today} % Leave empty to omit a date
\date{\DTMusedate{thedate}}
\renewcommand{\maketitlehookd}{%
    \begin{abstract}
        \noindent We define here an algorithm for indexing the system based on identifiers that are hashed values which is at the same time very 
        powerful to the writing and the reading. We call the Treee\texttrademark.
    \end{abstract}
}

%----------------------------------------------------------------------------------------

\begin{document}

% Print the title
\maketitle

%----------------------------------------------------------------------------------------
%	ARTICLE CONTENTS
%----------------------------------------------------------------------------------------

\section{Introduction}

\lettrine[nindent=0em,lines=3]{T}he challenge was to set up a powerful yet safe search engine to use when the data is some linked list of items that 
could be themselves connected to each other in subchains, indexed through their identifiers that are only made of hashed values (like SHA-256 string 
representations), and all stored in an immutable file.

Its best application is for a blockchain file where an \emph{item} is a transaction embedding a smart contract, and each subchain of items the 
subsequent uses and/or modifications of this smart contract.

This present document describes such indexing engine.

\begin{definition}[Item]
    \label{item}
    An \emph{item} is the actual object recorded in the immutable file subject to indexing.

    As previously stated, it is initially meant to be a transaction or a block in a blockchain file.
\end{definition}

\begin{definition}[Leaf]
    \label{leaf}
    A leaf $\lambda$ embeds the information helping to retrieve one or more possibly linked \emph{items}.
\end{definition}

%----------------------------------------------------------------------------------------

\section{Formal Description}

\subsection{Acyclic graph}

Treee\texttrademark~is an algorithm for indexing \emph{items} recorded in an immutable file based on their identifiers that are hashed values.

It is constructed as an acyclic graph (a tree $T$), each node containing either a node address (its sons) or a set of \emph{leaf}.

\begin{definition}[Hashed Identifier]
    \label{hashedIdentifier}
    We call $\iota$ a hashed identifier (or hashed value) the hexadecimal string representation of the result of a data $d$ passed through a hashing 
    function $\mathfrak{h}()$ such as:
    \begin{equation}
        \label{eq:hashedIdentifier}
        \iota := \mathfrak{h}(d)
    \end{equation}

    The passed data $d$ could be anything but it must be unique if it were to be used as identifier per se.
    
    In blockchains we operate, this data $d$ is usually the \emph{item} itself, ie. a transaction or a block.

    The hashing function could use any cryptographic hashing algorithm as long as it is set beforehand and once and for all in the targeted system
    \footnote{We currently use the \texttt{SHA-256} algorithm because of its wide adoption in both end-user and back-end environments}.
    The number $\mathfrak{N}$ represents the number of bits of the returned hash, eg. $256$ for \texttt{SHA-256}.
\end{definition}

The number of sons of a node is deterministic and depends on the depth of the tree.

Let $p_k$ be the number of sons of a node $N_k$ at depth $k$.

The goal is to create a balanced tree whose width is adaptive to decrease depth and optimize performance.

We are looking to index numbers, in this case the numerical value of the \emph{item}'s unique identifiers $\iota_i (\forall i \in \mathbb{N})$.

\begin{recall}
    An identifier is at its core a hashed value, that is its digest is fundamentally a byte array that could represent any positive integer.
\end{recall}


Lorem ipsum ...

%\vfill\eject % To force break column if need be
%\tableofcontents % Uncomment to add a table of contents

%----------------------------------------------------------------------------------------
%	REFERENCE LIST
%----------------------------------------------------------------------------------------

\begin{thebibliography}{99} % Bibliography

\bibitem[1]{feistel:hf}
Horst Feistel. \emph{Cryptography and Computer Privacy}, Scientific American, 1973.

\bibitem[2]{Feistel:cyd}
Cyril Dever. \emph{Feistel Cipher with Hash Round Function}, 2021.
\\\small\url{https://github.com/cyrildever/feistel}

\bibitem[3]{PRNG}
E. Barker, W. Barker, W. Burr, W. Polk, M. Smid. \emph{Recommendation for Key Management}, NIST, 2012.

\end{thebibliography}

%----------------------------------------------------------------------------------------

\end{document}